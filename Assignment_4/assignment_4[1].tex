\documentclass[journal,12pt,twocolumn]{IEEEtran}
\usepackage[utf8]{inputenc}
 
\usepackage{tfrupee}
\usepackage{enumitem}
\usepackage{amsmath}
\usepackage{amssymb}
\usepackage{graphicx}

\providecommand{\sbrak}[1]{\ensuremath{{}\left[#1\right]}}
\providecommand{\lsbrak}[1]{\ensuremath{{}\left[#1\right.}}
\providecommand{\rsbrak}[1]{\ensuremath{{}\left.#1\right]}}
\providecommand{\brak}[1]{\ensuremath{\left(#1\right)}}
\providecommand{\lbrak}[1]{\ensuremath{\left(#1\right.}}
\providecommand{\rbrak}[1]{\ensuremath{\left.#1\right)}}
\providecommand{\cbrak}[1]{\ensuremath{\left\{#1\right\}}}
\providecommand{\lcbrak}[1]{\ensuremath{\left\{#1\right.}}
\providecommand{\rcbrak}[1]{\ensuremath{\left.#1\right\}}}

\begin{document}

\newcommand{\myvec}[1]{\ensuremath{\begin{pmatrix}#1\end{pmatrix}}}
\let\vec\mathbf

\title{Assignment 4}
\author{\textbf{Himanshu Kumar Gupta (AI21BTECH11012)}}
\maketitle
\date {May 2022}

\textbf{\textit{Example 24 , class $12^{th}$ , CBSE:}}

Two cards are drawn successively with replacement from a well-shuffled
deck of 52 cards. Find the probability distribution of the number of aces.

\textbf{\textit{Solution:}}

No. of aces is random variable.

So,let the random variable $X\in{\cbrak{0,1,2}}$ denote
the number of aces in the card drawing experiment.

We know that,
\begin{align}
\label{eq:1}
\Pr\brak{X = i} = \frac{n\brak{X = i}}{\sum_{i=0}^2 n\brak{X = i}}
\end{align}
where $i\in\cbrak{{0,1,2}}$ and $n\brak{X = i}$ is the frequency of getting i ace

Now, there are 52 cards and it's drawn 2 times. So,total cases would be $52\times52$.

So,
\begin{align}
\label{eq:2}
\sum_{i=0}^2 n\brak{X = i}&=52\times52 \nonumber \\
&=2704
\end{align}

For X=2,

There are total 4 aces in deck of 52 cards and it's drawn 2 times and we need both as ace. So,total cases would be $4\times4$.

So,
\begin{align}
\label{eq:3}
n\brak{X = 2}&=4\times4     \nonumber  \\
&=16
\end{align}

For X=1,

There are 4 aces and 48 non-ace cards in deck of 52 cards and it's drawn 2 times and we need exactly 1 ace. So there are 2 possibility,first is that first card is ace and second one is not and the other case is first card is non-ace but other one is ace. So,total cases for this condition would be $4\times48+48\times4$.

So,
\begin{align}
\label{eq:4}
n\brak{X = 1}&=4\times48+48\times4        \nonumber \\
&=384
\end{align}

For X=0,

There are 48 non-ace cards in deck of 52 cards and it's drawn 2 times.So,for being both non-ace card the number of cases would be $48\times48$. 

So,
\begin{align}
\label{eq:5}
n\brak{X = 0}&=48\times48              \nonumber\\
&=2304
\end{align}

Now,

From equation \eqref{eq:1}, probability of getting 2 aces,
\begin{align}
\Pr\brak{X = 2} = \frac{n\brak{X = 2}}{\sum_{i=0}^2 n\brak{X = i}}
\end{align}
putting values from equations \eqref{eq:2} and \eqref{eq:3},
\begin{align}
\Pr\brak{X = 2}&=\frac{16}{2704}  \\
&=\textbf{.005917}
\end{align}
again,

from equation \eqref{eq:1},probability of getting 1 ace,
\begin{align}
\Pr\brak{X = 1} = \frac{n\brak{X = 1}}{\sum_{i=0}^2 n\brak{X = i}}
\end{align}

putting values from equations \eqref{eq:2} and \eqref{eq:4},
\begin{align}
\Pr\brak{X = 1}&=\frac{384}{2704}  \\
&=\textbf{.142}
\end{align}
again,

from equation \eqref{eq:1},probability of getting 0 ace,
\begin{align}
\Pr\brak{X = 0} = \frac{n\brak{X = 0}}{\sum_{i=0}^2 n\brak{X = i}}
\end{align}

putting values from equations \eqref{eq:2} and \eqref{eq:5},
\begin{align}
\Pr\brak{X = 0}&=\frac{2304}{2704}  \\
&=\textbf{.852}
\end{align}

	\begin{table}[ht!]
		\begin{tabular}{|l|c|c|c|}

\hline
\textbf{i}  & 0 & 1 & 2  \\
\hline
\textbf{Pr\brak{X=i}} & 0.852 & 0.142 & 0.005917\\
\hline
\end{tabular}

		\centering
		\vspace{5pt}
		\caption{}
		\label{table:1}	
	\end{table}
\end{document}